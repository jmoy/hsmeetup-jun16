
\documentclass{beamer}
\usetheme{CambridgeUS}
\usepackage{minted}

\title{Efficient Numerical Code in Haskell}
\author{Jyotirmoy Bhattacharya}
\institute[AUD]{Ambedkar University Delhi}
\date[Jun '16]{Delhi NCR Haskell Meetup, June~2016}
\begin{document}

\begin{frame}
  \maketitle
\end{frame}

\begin{frame}[fragile]
  \frametitle{About}
  \begin{itemize}
  \item Email: \verb+jyotirmoy@jyotirmoy.net+
  \item Website: \url{https://www.jyotirmoy.net}
  \item GitHub: \url{https://github.com/jmoy}
  \end{itemize}
\end{frame}

\begin{frame}[fragile=singleslide]
  \frametitle{Lists: The Good}
  \begin{block}{The List datatype}
    \begin{minted}{haskell}
data List a = Empty
            | Node a (List a)
    \end{minted}
  \end{block}
    \begin{itemize}
    \item Lazy links can represent infinite and
      self-referential structures:
      \begin{minted}{haskell}
fib = 1:1:zipWith (+) fib (tail fib)
      \end{minted}
    \item Lazy data can support modularity:
      \begin{minted}{haskell}
filter (even . fst) $ map (\i->(i,i^2)) [1..]       
      \end{minted}
    \item Updates can share representation with old values:
      \begin{minted}{haskell}
x = "cake"
y = 'b':tail x
\end{minted}
    \end{itemize}
\end{frame}

\begin{frame}
  \frametitle{Lists: The Bad}
  \begin{itemize}
  \item Access to the $n$-th elements in $O(n)$.
  \item High memory overhead and poor data locality.
  \item Pointer chasing for traversal.
  \item Data and links must be forced before use.
  \end{itemize}
\end{frame}

\begin{frame}[fragile]
  \frametitle{Sieve of Eratosthenes (Python)}
\begin{minted}{python}
import numpy as np
def count_primes(n):
  "Count no. of primes smaller than n"
  v = np.full(n,True,np.bool)
  # We want to set v[i] = False for non-primes
  v[0] = v[1] = False

  for p in range(n):
    if v[p]:
      # Found a prime
      # Mark all its multiples as non-prime
      for k in range(p*p,n,p):
        v[k] = False
  return np.count_nonzero(v)
\end{minted}
\end{frame}

\begin{frame}[fragile]{The \texttt{vector} package}
  \begin{itemize}
  \item Module \mintinline{haskell}|Data.Vector|.
    Array of heap-allocated values.
  \item Module \mintinline{haskell}|Data.Vector.Unboxed|.
    Array of unboxed values.
  \item Module \mintinline{haskell}|Data.Vector.Storable|.
    Array allocated in memory that can be passed to foreign functions.
  \end{itemize}
\end{frame}
\begin{frame}[fragile]
  \frametitle{Sieve of Eratosthenes (Haskell)}
  \begin{minted}{haskell}
import qualified Data.Vector.Unboxed as V
import Data.Vector.Unboxed ((//),(!))
countPrimes n = V.length $ V.filter id $ loop 2 mask0
  where
    mask0 = (V.replicate n True) // [(0,False),(1,False)]
    loop p mask 
      | p == n = mask
      | not (mask ! p) = loop (p+1) mask
      | otherwise = loop (p+1) mask'
        where
          mask' = mask // [(k,False)|k<-[p*p,p*p+p..(n-1)]]

  \end{minted}
\end{frame}

\begin{frame}[fragile]
  \frametitle{The \texttt{ST} Monad}
  \begin{itemize}
  \item A value of type \mintinline{haskell}|(ST s a)| is a computation which
    transforms a state of type \mintinline{haskell}|s| and
    additionally produces a value of type
    \mintinline{haskell}|a|.
  \item The state is not accessible to library users. So
    they cannot duplicate it.
  \item Action \mintinline{haskell}|newSTRef| creates a new
    \emph{mutable variable} within a state to which values
    can be written through \mintinline{haskell}|writeSTRef|
    and \mintinline{haskell}|modifySTRef| and read through
    \mintinline{haskell}|readSTRef|.
   \item The
    \mintinline{haskell}|runST| function takes a computation
    in \mintinline{haskell}|ST| and gives us the value
    produced. This is what allows us to embed
    \mintinline{haskell}|ST| computation in pure programs.
  \end{itemize}
\end{frame}

\begin{frame}[fragile]
  \frametitle{Stupid \texttt{ST} example}
\begin{minted}{haskell}
import Control.Monad.ST
import Data.STRef
import Control.Monad (forM_)

lengthJB::[a]->Int
lengthJB xs = runST $ do
  ctr <- newSTRef 0
  forM_ xs $ \_ -> modifySTRef ctr (+ 1)
  readSTRef ctr
\end{minted}
\end{frame}

\begin{frame}[fragile]
  \frametitle{Simulating a bank}
  \begin{block}{The External Interface}
    \begin{minted}{haskell}
type Name = String
type Ledger = [(Name,Double)]
data Transaction = Transaction
                      Name    -- From
                      Name    -- To
                      Double  -- Amount 

-- How do we implement this efficiently?
simulateBank::Ledger->[Transaction]->Ledger 
    \end{minted}
  \end{block}
\end{frame}
  
\begin{frame}[fragile]
    \frametitle{Simulating a bank (contd.)}
    \begin{minted}{haskell}
-- The state of a bank
type Bank s = HashMap Name (STRef s Double)
-- Create a bank reflecting a given ledger
mkBank :: [(Name,Double)] -> ST s (Bank s)
-- Modify the state of a bank to reflect a transaction
transact :: Bank s -> Transaction -> ST s ()
-- Turn the state of the bank into a ledger
readLedger :: Bank s -> ST s Ledger

simulateBank::Ledger -> [Transaction] -> Ledger
simulateBank ledger ts = runST $ do
  b <- mkBank ledger
  mapM_ (transact b) ts
  readLedger b

\end{minted}
\end{frame}

\begin{frame}[fragile]
    \frametitle{Simulating a bank (contd.)}
    \begin{minted}{haskell}
mkBank ledger = do
  let (n,as) = unzip ledger
  vs <- mapM newSTRef as
  return $ fromList $ zip n vs

transact b (Transaction cFrom cTo amt) = do
      modifySTRef' (b ! cFrom) (subtract amt)
      modifySTRef' (b ! cTo) (+ amt)

readLedger b = do
  let (n,v) = unzip $ toList b
  as <- mapM readSTRef v
  return $ zip n as
  \end{minted}
\end{frame}

\begin{frame}[fragile]
  \frametitle{The \texttt{ST} Monad (continued)}
      \begin{itemize}
      \item The type of \mintinline{haskell}|runST| is
    \begin{minted}{haskell}
runST :: (forall s. ST s a) -> a
    \end{minted}
    This is so that references to mutable variables cannot
    escape a \mintinline{haskell}|runST| and the computation
    encapsulated by it really is pure.

    \item Neither 
      \begin{minted}{haskell}
runST $ newSTRef 0
      \end{minted}
      nor 
      \begin{minted}{haskell}
\p -> runST $ readSTRef p
      \end{minted}
      are accepted by the typechecker.
      \end{itemize}
\end{frame}

\begin{frame}[fragile]
  \frametitle{Mutable Vectors}
  \begin{itemize}
  \item The \texttt{vector} package provides mutable vectors
    (\mintinline{haskell}|MVector|) which can be created,
    accessed and destructively modified in the \mintinline{haskell}|ST|
    monad (also variants for the \mintinline{haskell}|IO| monad).
  \item Like immutable vectors there are unboxed and
    storable variants.
  \item Conversion between mutable and immutable vectors is
    possible, but may copy the vector
    \begin{minted}{haskell}
freeze :: MVector s a -> ST s (Vector a) 
thaw :: Vector a -> ST s (MVector s a)
    \end{minted}
  \end{itemize}
\end{frame}
\begin{frame}[fragile]
  \frametitle{Sieve of Eratosthenes (again)}
{\small
  \begin{minted}{haskell}
import qualified Data.Vector.Unboxed as VI
import qualified Data.Vector.Unboxed.Mutable as V
countPrimes::Int->Int
countPrimes n
  = VI.length $ VI.filter id $ mask
  where
    mask = VI.create mkMask
    mkMask::ST s (V.MVector s Bool)
    mkMask = do
      v <- V.replicate n True
      V.write v 0 False
      V.write v 1 False
      forM_ [2..(n-1)] $ \p -> do
        b <- V.read v p
        when b $
          forM_ [p*p,p*p+p..(n-1)] $ \i ->
            V.write v i False
      return v
  \end{minted}
}
\end{frame}
\begin{frame}[fragile]
  \frametitle{REPA: REgular PArallel arrays}
  \begin{itemize}
  \item Key data type
    \begin{minted}{haskell}
      Array r sh e
    \end{minted}
    \item \mintinline{haskell}|r| can be
      \mintinline{haskell}|D|, \mintinline{haskell}|U|
      \mintinline{haskell}|F| and others.
    \item \mintinline{haskell}|sh| is a type of the shape
      class \mintinline{haskell}|Shape|. Example types
      \begin{minted}{haskell}
        Z :. Int :. Int
      \end{minted}
      Example values
      \begin{minted}{haskell}
        Z :. 0 :. 1
      \end{minted}
    \item Moving from a delayed to manifest represenation is
      under the programmer's control:
      \mintinline{haskell}|computeP|,  \mintinline{haskell}|computeS|.
  \end{itemize}
\end{frame}

\begin{frame}[fragile]
\frametitle{Parallelization for free}
\small
\begin{verbatim}
ubuntu@ip-172-31-40-69:~/examples-repa⟫ /usr/bin/time \
./dist/build/examples-repa/examples-repa \
    80 20 Wave.jpg WaveEC2.jpg +RTS -N1
9.59user 0.11system 0:09.70elapsed 100%CPU 
(0avgtext+0avgdata 556080maxresident)k
0inputs+43824outputs (0major+2447minor)pagefaults 0swaps

ubuntu@ip-172-31-40-69:~/examples-repa⟫ /usr/bin/time \
    ./dist/build/examples-repa/examples-repa \
    80 20 Wave.jpg WaveEC2.jpg +RTS -N4
14.09user 0.14system 0:04.66elapsed 305%CPU 
(0avgtext+0avgdata 559488maxresident)k
0inputs+43824outputs (0major+2910minor)pagefaults 0swaps
\end{verbatim}
\end{frame}
\end{document}

% Local Variables:
% TeX-command-extra-options: "-shell-escape"
% End:
